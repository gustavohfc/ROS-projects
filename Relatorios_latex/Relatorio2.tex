\documentclass{llncs}

\usepackage{forest}
\usepackage[utf8]{inputenc}
\usepackage{listings}

\definecolor{folderbg}{RGB}{124,166,198}
\definecolor{folderborder}{RGB}{110,144,169}
\def\Size{4pt}
\tikzset{
      folder/.pic={
        \filldraw[draw=folderborder,top color=folderbg!50,bottom color=folderbg]
          (-1.05*\Size,0.2\Size+5pt) rectangle ++(.75*\Size,-0.2\Size-5pt);  
        \filldraw[draw=folderborder,top color=folderbg!50,bottom color=folderbg]
          (-1.15*\Size,-\Size) rectangle (1.15*\Size,\Size);
      }
    }





\begin{document}

\title{Relatório do Trabalho 2}
\author{Gustavo Henrique Fernandes Carvalho - 14/0021671}
\institute{Fundamentos Computacionais de Robótica - 2017/2 - Universidade de Brasília}
\maketitle


\section{Informações do pacote}
\subsection{Dependências}
Para executar a simulação do trabalho 2 não é necessário nenhuma dependência além das dependências originais do pacote fcr2017, definidas no arquivo \textit{package.xml}.


\subsection{Rodando a simulação}
Para rodar a simulação é necessário apenas executar o seguinte comando no terminal:
\begin{lstlisting}[language=bash]
	$ roslaunch fcr2017 trabalho2.launch
\end{lstlisting}
O arquivo \textit{trabalho2.launch} já inicia todos os nós necessários para a simulação, inclusive o nó de controle do trabalho 2.

A simulação é dividida em 3 etapas, na primeira etapa o Pioneer se move para uma posição inicial no mapa topológico definida pelo usuário, na segunda etapa o Pioneer percorre todo o mapa topológico preenchendo a grade de ocupação de cada região do mapa, e na última etapa o Pioneer vai para a posição final também  definida pelo usuário, as interfaces de entrada e saída da simulação são explicadas com mais detalhes na seção \ref{sec:io}.


\subsection{Arquivos}
A Figura \ref{file_tree} mostra os novos arquivos do pacote fcr2017 utilizados para a realização do trabalho 2, os arquivos originais do pacote não são mostrados.

\pagebreak
\begin{figure}[h!]
\begin{forest}
      for tree={
        font=\ttfamily,
        grow'=0,
        child anchor=west,
        parent anchor=south,
        anchor=west,
        calign=first,
        inner xsep=7pt,
        edge path={
          \noexpand\path [draw, \forestoption{edge}]
          (!u.south west) +(7.5pt,0) |- (.child anchor) pic {folder} \forestoption{edge label};
        },
        % style for your file node 
        file/.style={edge path={\noexpand\path [draw, \forestoption{edge}]
          (!u.south west) +(7.5pt,0) |- (.child anchor) \forestoption{edge label};},
          inner xsep=2pt,font=\scriptsize\ttfamily
                     },
        before typesetting nodes={
          if n=1
            {insert before={[,phantom]}}
            {}
        },
        fit=band,
        before computing xy={l=15pt},
      }  
	[fcr2017
		[launch
			[trabalho2.launch - Utilizado para iniciar a simulação, file]
			[trabalho2.rviz - Configuração do RViz, file]
			[..., file]
		]
		[map\_output - Diretório onde as imagens das grades de ocupação são salvas.
		]
		[src
			[140021671\_Trabalho2.cpp - Código fonte do trabalho 2, file]
			[common\_lib - Biblioteca implementada para facilitar o reuso e modificação do código
				[common.cpp - Implementação de funções gerais, file]
				[common.h - Definição de funções e estruturas gerais, file]
				[graph.cpp - Implementação das funções para manipulação de grafo, file]
				[graph.h - Definição das funções e estruturas para manipulação de grafo, file]
				[grid\_map.cpp - Implementação das funções que fazem o mapa de ocupação, file]
				[grid\_map.h - Definição das funções e estruturas que fazem o mapa de ocupação, file]
				[laser\_sensor.cpp - Implementação das funções que manipulam os dados do sensor laser, file]
				[laser\_sensor.h - Definição das funções e estruturas que manipulam os dados do sensor laser, file]
				[motion\_controller.cpp - Implementação das funções de controle de movimento, file]
				[motion\_controller.h - Definição das funções e estruturas de controle de movimento, file]
				[odometer.cpp - Implementação das funções manipulam os dados do odômetro, file]
				[odometer.h - Definição das funções e estruturas manipulam os dados do odômetro, file]
			]
		]
		[CIC\_graph.txt - Definição do mapa topológico do primeiro andar do prédio CIC/EST, file]
		[README.txt - Mais informações sobre o pacote, file]
		[..., file]
	]
\end{forest}
\caption{Estrutura dos arquivos.}
\label{file_tree}
\end{figure}


\section{Interfaces da simulação} \label{sec:io}
\subsection{Interface de entrada de dados}


\end{document}
